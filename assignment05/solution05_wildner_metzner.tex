\documentclass[10pt,a4paper]{scrartcl}
\usepackage{../uebungsblatt}
\usepackage{dsfont}
\usepackage[utf8]{inputenc}
\usepackage{ngerman}
\usepackage{enumitem}
\usepackage{stmaryrd}
\usepackage{a4wide}
\usepackage[ruled, vlined]{algorithm2e}
\usepackage{amsmath}
\usepackage{dsfont}
\usepackage{tikz}
\usepackage{tikz-qtree}
\usepackage{float}
\usepackage{xcolor}
\usepackage{listings}

\lstset{
  moredelim=[is][\underbar]{-}{-}
}

\newcommand{\Pot}{\mathcal{P}}

\newcommand{\N}{\mathbb{N}}
\newcommand{\Z}{\mathbb{Z}}
\newcommand{\R}{\mathbb{R}}
\newcommand{\der}{\,\mathrm{d}}


%Kopfzeile:
\fancyhead[R]{Seite: \thepage \hspace{0.5ex} von \pageref{LastPage}}
\fancyhead[L]{Wildner \& Metzner }

\begin{document}

\uebkopfzeile
  {Computer Vision and Image Analysis} % Titel der Veranstaltung
  {WS 14/15}  % Semesterangabe, Übungsgruppe
  {}    % Dozenten, Übungsleiter
  {Jens Metzner \& Manuel Wildner}    % Loesungsblattbearbeiter

\uebtitel
{Solution of assignment 5} % Titel (gross und zentriert)
{27. Nov} % Datum der Abgabe


\solution{5.1}{the SIFT paper}{10}
\begin{itemize}
	\item[(i)] $k$ is computed by $k = 2^\frac{1}{s}$ to get $s$ intervals for each octave of the scale space. So $k = \sqrt{2}$\\
	The precise scales are $\sigma,\ 1.41*\sigma,\ \sigma,\ 2.83*\sigma$ and $4*\sigma$
	\item[(ii)]
	\item[(iii)] It is a approximation of the Hessian and derivative of D (scale-space function) by using differences of neighboring sample points.
	\item[(iv)]
	\item[(v)] In the figure, they use a 2x2x8 = 32 feature vector and in their experiments, a 4x4x8 = 128 feature vector.
\end{itemize}

\solution{5.2}{blob detection using Difference of Gaussians in scale space}{20}
$\rightarrow$ See code : )

\end{document}