\documentclass[10pt,a4paper]{scrartcl}
\usepackage{../uebungsblatt}
\usepackage{dsfont}
\usepackage[utf8]{inputenc}
\usepackage{ngerman}
\usepackage{enumitem}
\usepackage{stmaryrd}
\usepackage{a4wide}
\usepackage[ruled, vlined]{algorithm2e}
\usepackage{amsmath}
\usepackage{dsfont}
\usepackage{tikz}
\usepackage{tikz-qtree}
\usepackage{float}
\usepackage{xcolor}
\usepackage{listings}

\lstset{
  moredelim=[is][\underbar]{-}{-}
}

\newcommand{\Pot}{\mathcal{P}}

\newcommand{\N}{\mathbb{N}}
\newcommand{\Z}{\mathbb{Z}}
\newcommand{\R}{\mathbb{R}}
\newcommand{\der}{\,\mathrm{d}}


%Kopfzeile:
\fancyhead[R]{Seite: \thepage \hspace{0.5ex} von \pageref{LastPage}}
\fancyhead[L]{Wildner \& Metzner }

\begin{document}

\uebkopfzeile
  {Computer Vision and Image Analysis} % Titel der Veranstaltung
  {WS 14/15}  % Semesterangabe, Übungsgruppe
  {}    % Dozenten, Übungsleiter
  {Jens Metzner \& Manuel Wildner}    % Loesungsblattbearbeiter

\uebtitel
{Solution of assignment 4} % Titel (gross und zentriert)
{11. Nov} % Datum der Abgabe


\solution{4.1}{the Fourier transform with pen and paper}{10}

\begin{align*}
f(x,y) &= \left\{\begin{array}{cl} 1, & \mbox{if }0 \leq x \leq \frac{1}{2} \mbox{ and } -\frac{1}{4} \leq y \leq \frac{1}{4},\\ 0, & \mbox{otherwise} \end{array}\right.\\
%
\hat f(w_x,w_y) &= \int_\R \int_\R f(x,y) \exp{\left(-2\pi i(w_xx+w_yy)\right)}\der x \der y\\
\end{align*}
Due to the case distinction of $f(x,y)$, we can limit the integrals and replace the function with $1$. Then we simplify the integrals:
\begin{align*}
\hat f(w_x,w_y) &= \int_{-\frac{1}{4}}^\frac{1}{4} \int_0^\frac{1}{2} 1 \exp{\left(-2\pi i(w_xx+w_yy)\right)}\der x \der y\\
%
\hat f(w_x,w_y) &= \int_{-\frac{1}{4}}^\frac{1}{4} \left(\int_0^\frac{1}{2} \exp{\left(-2\pi i(w_xx+w_yy)\right)}\der x \right)\der y\\
%
\hat f(w_x,w_y) &= \int_{-\frac{1}{4}}^\frac{1}{4} \left(\frac{i\exp{\left(-i\pi (w_x+2w_yy)\right)}}{2\pi w_x} - \frac{i\exp{\left(-i\pi w_x-i\pi(w_x+2w_yy)\right)}}{2\pi w_x}\right)\der y\\
%
\hat f(w_x,w_y) &= \frac{\sin{\left(\frac{\pi w_y}{2}\right)} \left(\sin{(\pi w_x)} + i \cos{(\pi w_x)} - i\right)}{2 \pi^2 w_x w_y}\\
\end{align*}
We use this formular and get following results and also amplitude as $\log|\hat f|$ and phase as $\arg{\hat f}$:
\begin{figure}[h]
\center
\renewcommand\arraystretch{1.5}%
\begin{tabular}{l|ccc}
 & result & $\log|\hat f|$ & $\arg{\hat f}$\\\hline
$\hat f(1,0)$ & $-\frac{i}{2\pi}$ & $\log\frac{1}{2\pi} = -1,838$ & $-\frac{\pi}{2}$ \\
$\hat f(0,1)$ & $\frac{1}{2\pi}$ & $\log\frac{1}{2\pi} = -1,838$ & 0 \\
$\hat f(-1,0)$ & $\frac{i}{2\pi}$ & $\log\frac{1}{2\pi} = -1,838$ & $\frac{\pi}{2}$ \\
$\hat f(0,-1)$ & $\frac{1}{2\pi}$ & $\log\frac{1}{2\pi} = -1,838$ & 0 \\
\end{tabular}
\end{figure}

\solution{4.2}{the Fourier transform in Matlab}{15}

\begin{itemize}
\item[\textbf{(b)}] The amount of shifted pixels depends on the size of the kernel.
\item[\textbf{(c)}] It would apply noise because the values of the outer regions of the Fourier transformation of the central difference kernel are a lot higher compared to the derivate of the Gaussian kernel, which means, that some unimportant frequencies are pushed.
\end{itemize}


\solution{4.3}{hybrid images}{5}

\end{document}